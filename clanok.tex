% Predbežná verzia článku
% Marek Adamove, AIS ID: 110758
% Štvrtok 13:00

\documentclass[10pt,slovak,a4paper]{article}

\usepackage[slovak]{babel} 	% SK jazyk
\usepackage[IL2]{fontenc} 		% SK jazyk
\usepackage[utf8]{inputenc}	% SK jazyk
\usepackage{graphicx} 			% Obrázky
\usepackage{url} 				% Formátovanie URL ( \url )
\usepackage{hyperref} 			% Aktívne odkazy v texte (!! niektoré posúvajú text !!)
\usepackage{cite} 				% Citácie


\pagestyle{headings}

\title{Klady a zápory dištančnej výučby v komparácii s prezenčnou výučbou\thanks{Semestrálny projekt v predmete Metódy inžinierskej práce, ak. rok 2020/21, vedenie: Ing. Michal Hatala, PhD.}}  
% \thanks --> referencia pod čiarou na konci strany

\author{Marek Adamove\\[2pt]
	{\small Slovenská technická univerzita v Bratislave}\\
	{\small Fakulta informatiky a informačných technológií}\\
	{\small \texttt{xadamove@stuba.sk}}
	}

\date{\small 20. október 2020} 



\begin{document}

\maketitle

\begin{abstract}
Online vzdelávanie je relatívne novou a dôležitou súčasťou vzdelávacieho procesu. V dnešnej
dobe nám internet a mnohé kancelárske balíky uľahčujú prácu a ponúkajú možnosti zdieľania informácií
s ostatnými bez toho, aby sme boli s danou osobou v styku v realite. To, že je vývoj takéhoto charakteru vyučovania dôležitý je aj fakt aktuálnej pandémie COVID-19, kvôli ktorej sa stáva práve dištančná metóda vzdelávania čoraz viacej bežnou. Článok analyzuje aktuálny stav v danej oblasti, bližšie opisuje a porovnáva klady a zápory dištančnej formy vzdelávania v porovnaní s prezenčnou formou, ale venuje sa aj možným dôsledkom, ktoré môže daná forma vzdelávania spôsobiť. V závere sa nachádza sumarizácia kladov, záporov a dôsledkov, ktoré so sebou dištančná metóda prináša a z nich vyvodený záverečný úsudok.
\end{abstract}



\section{Úvod}

	V modernom svete sa dištančná forma výučby pomaly ale určite stáva čoraz viacej využívanou metódou vzdelávania. Významní profesori Dr. Michael Grahame Moore a Greg Kearsley definujú dištančnú formu vzdelávania ako poskytovanie plánovaného vzdelávania študentom oddelených od svojich inštruktorov geograficky alebo časovo.\cite{moore_kearsley_2012} Z tohto teda vyplýva, že je to každá edukačná činnosť, ktorá sa deje medzi osobami od seba zemepisne či časovo oddelených. Takáto forma vzdelávania so sebou prináša určite klady, rovnako avšak aj zápory. V nasledujúcom článku sa venujem dištančnej forme vzdelávania, jej histórii či kladom a záporom oproti klasickej prezenčnej forme vzdelávania.




\section{Klady dištančnej formy vzdelávania}
	Dištančná forma vzdelávania priniesla revolúciu vo vyučovacom procese. Nesie so sebou jednoznačné výhody, ktoré uľahčujú vzdelávanie vo viacerých smeroch. Zdieľanie informácií a poznatkov bez toho, aby bol potrebný kontakt v realite medzi účastníkmi je esenciálnou zásadou tejto formy výučby. Táto revolúcia umožňuje mnohým ľuďom dostať sa k vzdelaniu bez toho, aby boli reálne prítomní v škole. Tento fakt naznačuje mnohé klady, ale určite aj zápory takéhoto typu vzdelávania. Klady dištančnej formy vzdelávania by som rozdelil do 5 kategórií:

\subsection{Vhodnosť}
	Skutočnosť, že dištančné vzdelávanie spočíva v edukačnej činnosti medzi geograficky či časovo oddelenými osobami nám umožňuje upustiť od cestovania do školy a navštevovať tak kľudne aj školu na opačnej strane Zeme z pohodlia domova. Významným kladom je taktiež jednoduchší spôsob pripravenia prednášky či vyučovacej hodiny s hosťom, keďže všetky prítomné osoby nemusia byť spolu napríklad v jednej miestnosti. Znamená to aj jednoduchšiu realizáciu diskusie medzi viacerými expertmi v danej oblasti širokej verejnosti.

	Pre študentov, ktorým robí problém ústne vyjadrovanie môže pomôcť práve skutočnosť, že v dištančnej forme vzdelávania môže byť dominantnou práve písomná komunikácia prostredníctvom napríklad e-mailu. Týmto spôsobom majú študenti s introvertnou osobnosťou možnosť lepšie pochopiť preberanú látku, nakoľko môžu cítiť menší psychický nátlak. Ďalšou výhodou, ktorá stojí za zmienku je možnosť ľahšej spätnej väzby, nakoľko v online priestore sa študenti cítia viac anonymne a preto ľahšie vyjadria svoj názor k určitej tematike.

\subsection{Vzťahy medzi účastníkmi vzdelávania}
	Väčšina univerzít má veľký počet zahraničných študentov či zamestnancov, čo spôsobuje vysokú mieru diverzity medzi študentmi a zamestnancami. Táto rôznorodosť môže vyvolávať konflikty medzi účastníkmi vzdelávacieho procesu a tak obmedziť kvalitu vzdelania. Rasizmus je najčastejšia forma diskriminácie s ktorou sa zahraniční študenti a zamestnanci stretávajú. Podľa štatistiky z univerzity v Yale, 40\% študentov inej ako prevládajúcej bielej rasy zažilo na území kampusu univerzity v Yale istú formu rasovej diskriminácie. Najväčšiu mieru rasizmu zažili študenti čiernej rasy, najmenej hispánskej rasy, najčastejšími aktérmi boli ostatní študenti, ale niektorí sa stretli aj s diskrimináciou zo strany členov fakulty.\cite{david_shimer_2015}

 	Vyššia miera anonymity v online priestore a menej častá interakcia medzi účastníkmi vyučovania môže dopomôcť k zníženiu alebo úplnému odstráneniu diskriminácie medzi študentmi na základe rasy, pohlavia, sexuálnej orientácie, viery, národnosti, veku atď. Zníženie rozsahu diskriminácie napomáha k optimistickejšej nálade a teda nedomotivuje zahraničných študentov či učiteľov vo svojom štúdiu či práci.

\subsection{Súčinnosť študentov}
	Menej sociálneho kontaktu alebo až jeho úplné vypustenie medzi študentmi si vyžaduje väčšiu snahu pri skontaktovaní osôb s ktorými potrebujeme pracovať spoločne. Skupinové zadania nie sú neobvyklé a teda vedieť správne kontaktovať so svojou skupinou je esenciálne pre úspešné zvládnutie úlohy. V princípoch férovosti medzi  si študenti skupinové zadanie medzi sebou rozdelia na menšie úlohy, ktoré má každý vypracovať. Takto každý člen skupiny vykoná relatívne rovnaký podiel práce a skupinový projekt teda nie je prezentovaný hlavne najviac spoločenským človekom. \cite{kumar2010pros}

	Pri dištančnej forme vzdelávania je potrebná častá interakcie medzi účastníkmi, aby sa predišlo prípadným nedorozumeniam či chybám. V online priestore je možnosť zdieľať svoje nápady a prípadnú konštruktívnu kritiku pre mnohých študentov oveľa jednoduchšie ako v realite. Väčšie množstvo spätnej väzby môže priniesť na konci kvalitnejší výsledok

\subsection{Ekonomické výdaje}
	V krajinách, v ktorých nie je štúdium na univerzite hradené štátom sú na univerzitách ponúkané študijné programy realizované dištančnou metódou väčšinou lacnejšie ako tie, ktoré sú realizované klasickou prezenčnou metódou. Toto umožňuje mnohým ľuďom aj zo slabších pomerov dostať sa ku kvalitnému a cennému vzdelaniu. Možnosť študovať z domu taktiež dokáže znížiť výdaje v cestovaní ale aj v ubytovaní a s ním spojených výdavkov ako strava, ktoré by boli potrebné v prípade, kedy je domov študenta vo väčšej vzdialenosti od školy. 

\subsection{Ostatné výhody}
	Pri dištančnej forme vzdelávania sa taktiež rozvíja individualita a analytické myslenie študentov, pretože kvôli menšiemu kontaktu s učiteľom je často študent donútený sám hľadať súvislosti medzi zadanými faktami. Zároveň motivuje študenta vytvoriť si pravidelnú učebnú rutinu, pristupovať k štúdiu zodpovedne a zdokonaliť svoje schopnosti v oblasti time managementu. Taktiež pri online vzdelávaní si ako študenti tak aj učitelia vylepšujú svoje technické schopnosti, nakoľko je potrebná práca s počítačom a rôznymi aplikáciami na realizáciu dištančnej výučby. Istou výhodou je aj realizácia online testov, ktoré študentovi poskytnú výsledok hneď po jeho odoslaní čo uľahčuje prácu učiteľom. 

\section{Zápory dištančnej formy vzdelávania}
	Tak ako revolúcia vo forme dištančnej formy vzdelávania priniesla mnohé klady, priniesla aj možné zápory. Tieto problémy často zhoršujú podmienky študentov a učiteľov či už pri komunikácii medzi účastníkmi alebo aj pri samotnom procese vyučovania. Práve preto je potrebné zvážiť všetky prekážky s ktorými sa môžeme stretnúť, aby sme predišli nechceným komplikáciám. Zápory dištančnej formy vzdelávania rozdelím do 4 kategórií:

\subsection{Obmedzený sociálny kontakt}
	Nakoľko pri dištančnej forme vyučovania sú v dnešnej dobe využívané hlavne video-konferenčné či iné online služby, sociálny kontakt v realite s ostatnými účastníkmi nie je potrebný. To vedie k poklesu komunikácie medzi ľuďmi na minimálne množstvo čo sťažuje vytvorenie priateľstiev medzi spolužiakmi. Život mimo svojich spolužiakov môže študentovi privodiť pocity osamelosti a úzkosti, ktoré následne môžu viesť k zníženej produktivite. 

	Keďže veľká časť komunikácie medzi študentom a učiteľom sa odohráva prostredníctvom e-mailu je dôležité exaktné vyjadrovanie, pretože komunikácia prostredníctvom e-mailu nie je taká spontánna a časovo rýchla ako komunikácia v realite. Pri nepresnom opísaní problému môže dôjsť k jeho nepochopeniu zo strany učiteľa a tak sa komunikácia a celkovo doba trvania problému predĺži. 

	K zníženej interakcii môže dôjsť aj napríklad počas online prednášky, keďže je ťažšie nadviazať osobný kontakt medzi učiteľom a študentom. Sú aj prípady, kedy počas online prednášky nie je dovolené do nich zasahovať a tak na odpovede k svojim otázkam musí študent prísť sám alebo kontaktovať učiteľa mimo vyučovania. Takéto vedenie vyučovacích hodín online môže viesť k zníženiu kvality vzdelania a nespokojnosti zo strany študentov.

\subsection{Technika}
	K novodobej podobe dištančnej formy vzdelávania je nutné určité technické vybavenie ako počítač a pripojenie na internet, prípadne ešte aj webkamera či mikrofón. Toto vybavenie zväčša stojí väčšie množstvo peňazí a teda nie každý si ho môže dovoliť. 

	Práca s počítačom si vyžaduje aj určité technické zručnosti, v ktorých pokiaľ študent alebo učiteľ má nedostatky môže dôjsť ku komplikáciám. Je časté, že staršia generácia je menej skúsená s manipuláciou počítača a teda má obmedzené svoje možnosti realizovania sa prostredníctvom neho či podobnej techniky.

	Ďalším záporom z technickej stránky je jej nevyspytateľnosť. Keďže väčšina tohto vybavenia je závislého na pripojení k elektrike, náhly výpadok elektrického pripojenia môže spôsobiť nechcené komplikácie počas štúdia, ako napríklad stratu rozpísanej práce či podobne. To isté platí aj pre pripojenie na internet, jeho výpadok môže v čase napríklad online prednášky znemožniť prítomnosť na nej, kvôli čomu sa nám nemusí dostať výkladu učiva.

\subsection{Efektivita zadaní a testov}
	Aj počas štúdia dištančnou formou je potrebné preveriť vedomosti a zručnosti študentov pomocou testov a zadaní. Väčšina online testov je na báze výbere správnej možnosti z viacerých ponúkaných, pretože takéto formy testových úloh vie testovací systém vyhodnotiť sám. Úlohy s otvorenou odpoveďou sú menej časté, nakoľko tieto úlohy musia byť kontrolované osobne učiteľom. 

	Napriek tomu, že v dnešnej dobe už existujú viaceré aplikácie, ktorých cieľom je zamedziť podvádzanie pri testoch, stále je relatívne časté. Aj tento problém núti mnohých učiteľom vymýšľať testové otázky tak, aby na internete nebolo tak jednoduché nájsť odpovede. Takto komplikovane zadané otázky nemusia byť ľahko pochopiteľné a tak sa môže stať, že i študent pripravený na test nemusí odpoveď poznať. Za dôsledok to môže mať horšie priemerne dosiahnuté hodnotenie. 

\subsection{Ostatné nevýhody}
	Pri realizácii vzdelávania dištančnou formou nie je neobvyklý chaos. Dochádzať k nemu môže napríklad pri prednáškach z rôznych predmetov, ktoré sú uskutočňované pomocou rozdielnych video-konferenčných služieb či pri zhromažďovaní materiálov z viacerých predmetov .

	Ďalšou veľkým záporom je fakt, že realizácia praxe pri štúdiu, teda použitie teórie v realite je prostredníctvom dištančnej formy vzdelávania len ťažko realizovateľná. Pri závažnosti tohto kladu je rozhodujúci práve aj odbor, ktorý študent študuje. Je jasné, že všeobecne pri viacej teoretických odboroch nemožnosť praxe nemá až taký veľký dopad ako pri odboroch praktických.

\section{Porovnanie dištančnej a prezenčnej formy výučby}
	Ako bolo v predchádzajúcich častiach ukázané, dištančná forma vzdelávania má jednoznačne svoje klady, na druhej strane ale aj zápory. Dištančná forma štúdia môže byť vhodná pre ľudí, ktorí si z určitého dôvodu ako napríklad nedostatok financií či veľká vzdialenosť od školy nemôžu dovoliť prezenčné štúdium. Prezenčné štúdium môže predstavovať veľké výdaje na bývanie v inom meste či státe, stravovanie či rôznych pomôcok potrebných ku štúdiu.

	Pri prezenčnej metóde výučby, kedy študent navštevuje prednášky na území univerzitného kampusu môže študent dostať kvalitnejšie vzdelanie a výklad, v porovnaní s prednáškou realizovanou pomocou video-konferencie. Prednášateľ v reálnom živote má možnosť študentov vidieť, a teda má lepšiu možnosť zistiť či študenti preberanej látke rozumejú. Klasické prezenčné prednášky taktiež majú možnosť byť viacej kreatívne a myslenie rozvíjajúce. V neposlednom rade, pozornosť a sústredenosť študentov pri výklade je v prípade prezenčnej metódy vyššia, ako tá pri metóde dištančnej. 

	Fyzická účasť na univerzite určite aj pomáha vytvoriť viaceré priateľstvá, nakoľko nadviazať konverzáciu v reálnom priestore je jednoduchšie a osobnejšie, ako to v online priestore. Študent v kampuse má viacej možností na strávenie svojho voľného času, či už s priateľmi alebo bez. Každodenný fyzický sociálny kontakt napomáha väčšine študentom k zdravej psychohygiene a optimistickejšiemu náhľadu na svet. Na druhej strane, pre introvertných študentov môže univerzitný kampus pripadať hektický a spôsobovať úzkostné stavy. Fyzická prítomnosť na univerzite môže avšak zapríčiniť väčšiu mieru diskriminácie v porovnaní s dištančným vzdelávaním, ktorá taktiež môže mať dopad na študentov celkový dojem zo štúdia. 

	Počas štúdia na univerzite na území kampusu si študent dokáže zvýšiť mieru motivácie oveľa jednoduchšie ako pri dištančnej výučbe z domu. Toto má za následok, že študent podáva lepší výkon a tým pádom získava lepšie študijné výsledky. Zároveň štúdium na území kampusu univerzity ponúka študentom možnosť využiť knižničné služby univerzity, ktoré sú pohodlnejším riešením hľadania odbornej literatúry. Pri prezenčnej výučbe sa avšak nerozvíja individualita a samostatnosť študenta až v takej miere ako pri dištančnej výučbe, kde je potrebné, aby si študent väčšiu časť informácií spracoval sám.

	Prezenčná výučba nevyžaduje také množstvo technického vybavenia, aké je potrebné pri dištančnej výučbe, tým pádom je organizácia vzdelávacieho procesu jednoduchšia. Spravidla na vyučovaní sa môže fyzicky zúčastniť viac študentov, nakoľko pri online video-konferenciách je zväčša obmedzený počet účastníkov, ktorý nemusí vyhovovať aktuálnemu počtu študentov zapísaných na danom predmete. Nevyspytateľnosť techniky ako výpadky elektrického prúdu nemusia spôsobiť také škody, čo znamená, že priebeh vyučovania nie je až tak ohrozený. Počítačová gramotnosť študentov či profesorov nie je až tak veľmi nutná.

	Pri fyzickej účasti študenta na mieste univerzity si profesori taktiež vedia lepšie odpozorovať aktivitu a účasť študentov počas skúšok. V porovnaní s dištančnou metódou písania testov si pri prezenčnej metóde môžu profesori dovoliť na testoch položiť aj otázky, na ktoré sa očakáva otvorená odpoveď, či nemusia zadania otázok komplikovať kvôli obavám z používania internetu za účelom podvádzania. Nevýhodou takýchto testov avšak býva dlhšia doba opravy, keďže musia byť opravené osobne danou zodpovednou osobou. 

	Vylepšenie praktických zručností pri ktorých asistuje učiteľ osobne je taktiež výhodou prezenčnej výučby, nakoľko pri dištančnej metóde je takáto možnosť minimálna. Dobré praktické zručnosti môžu absolventovi daného odboru pomôcť pri hľadaní zamestnania či pri všeobecnom zlepšovaní sa v danej oblasti. 

\section{Záver}
	Fakt, že vznik dištančnej formy vzdelávania znamená revolúciu v organizácii vzdelávania je nepopierateľný. Pre mnohých ľudí, záujemcov o štúdium, znamená výhodnejšiu alternatívu z osobných, ekonomických či iných dôvodov. Ponúka vysokú mieru flexibility, čo predstavuje vyššiu mieru voľnosti, ale aj samostatnosti a sebadisciplíny. Študent musí pri voľbe (ak má takú možnosť) zvažovať všetky klady a zápory dištančnej formy vzdelávania na základe svojich možností. Všeobecne sa preto nedá povedať, ktorá metóda vyučovania je efektívnejšia, nakoľko každá má svoje klady i zápory. 

	Avšak domnienka, že úplne nahradí prezenčnú metódu nie je až tak reálna. Záujem o prezenčnú formu vzdelávania bude vždy dominovať nad záujmom o formu dištančnú. Dôvodom môže byť aj skutočnosť, kedy sa mladí ľudia v čase svojho vysokoškolského štúdia chcú zoznámiť s mnohými rovesníkmi a tak vytvoriť priateľstvá aj na celý život. 


% týmto sa generuje zoznam literatúry z obsahu súboru literatura.bib podľa toho, na čo sa v článku odkazujete
\bibliographystyle{unsrt} % prípadne alpha, abbrv alebo hociktorý iný
\bibliography{zdroje}


\end{document}
